 \section{Market Opportunity}

 \par The structure of traditional banks in the maintenance of ledgers, settlement and clearing of payment transactions
 is too complicated, and high IT maintenance costs are incurred every year. As an advanced distributed database mechanism, 
 blockchain allows transparent sharing of information across enterprise networks. Blockchain databases store 
 data in blocks, and databases are linked together in a chain. The data is consistent in time because the chain 
 cannot be deleted or modified without network consensus. Thus, immutable ledgers can be created using 
 blockchain technology in order to track orders, payments, accounts and other transactions. This decentralized 
 approach provides great security and transparency and has enormous application prospects.

 \par The throughput of transactions has always been a constant concern of business personnel, because it will limit 
 the application scope of the business to a certain extent. Generally speaking, the throughput of the blockchain 
 determines the number of users that can be supported. At the same time, the speed of transactions is important 
 to both customer engagement and satisfaction. In order to improve throughput, the industry is constantly innovating 
 and carrying out various optimization reforms. Our project is based on \RCC{}, which can make better use of resources 
 and make the throughput reach the level of multiple replicas, which has far-reaching significance and commercial 
 prospects for the development of high-throughput resilient databases in the future.

 \par In the business world, an unstable system will lead to the loss of customers, and the stability of the system 
 plays a pivotal role in improving the stickiness of customers. Therefore, the design of the blockchain system should 
 minimize interruptions caused by Byzantine nodes and have better resilience. Concurrent consensus protocols like \RCC{} 
 are benificial for resisting interruptions caused by network attacks or software problems, preventing the loss of user data. 
 In terms of data preservation and reducing system maintenance costs, our project has broad commercial application 
 prospects.
