\section{Market Opportunity}
\par The blockchain adopts a decentralized distributed accounting method, which provides great security 
and transparency. The traceability and automated processing of data also enable the blockchain to 
bring more business benefits. In recent years, blockchain technology and industry have developed 
rapidly around the world \cite{bitcoin, ethereum, hotstuff}, and its applications have been extended 
to digital finance, the Internet of Things, intelligent manufacturing, supply chain management, digital 
asset trading, and other fields, showing broad application prospects \cite{gupta2021fault}.

\par In the blockchain world, a transaction is a basic unit that makes up a transaction. Transaction 
throughput can largely limit or expand the application scenarios of blockchain business. Currently, 
transaction throughput is a hot indicator for evaluating performance. The higher the throughput, 
the wider the scope of the application and the larger the user scale that the blockchain can support. 
The rapid transaction increases customer engagement and satisfaction rates. To improve throughput, 
the industry has put forward an endless stream of optimization solutions, all of which lead to the same 
goal. \RCC{} \cite{rcc} implemented in NexRes would push throughput beyond single-replica limit by better 
utilizing available resources, opening the door to the development of new high-throughput resilient 
databases and federated transaction processing systems.

\par In the presence of Byzantine participants \cite{lao2020survey}, resilience enables blockchains to 
reduce the likelihood of disruption and recover faster. In the business 
world, a non-stable system will hugely decrease the chances of converting the visitor into a customer 
and declines customer stickiness. Still, concurrent agreement protocol like \RCC{} promises increased 
resilience against failures, which can give strong support for data provenance, and facilitated federated 
data processing in a miscellaneous environment with numerous independent parties. Hence, once the system 
faces disruption caused by cyberattacks or software issues, consensus-based protocol prevents system damage 
and provides a safe haven for user data. The risk resilience and the feature of data quality preservation 
decrease system maintenance costs, which indicates great market prospects.
