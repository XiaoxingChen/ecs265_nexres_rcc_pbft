\section{Market Opportunity}
\par The blockchain adopts a decentralized distributed accounting method, which provides great security 
and transparency. The traceability and automated processing of data also enable the blockchain to 
bring more business benefits. In recent years, blockchain technology and industry have developed 
rapidly around the world\cite{gupta2021fault, bitcoin, ethereum, hotstuff}, and its applications have been extended to digital finance, Internet of 
Things, intelligent manufacturing, supply chain management, digital asset trading and other fields, 
showing broad application prospects.

\par In the blockchain world, a transaction is the basic unit that makes up a transaction. Transaction 
throughput can largely limit or expand the application scenarios of blockchain business. Currently, 
transaction throughput, is a hot indicator for evaluating performance. The higher the throughput, 
the wider the scope of application and the larger the user scale that the blockchain can support. 
Rapid transaction increases customer engagement and satisfaction rates. In order to improve throughput, 
the industry has put forward an endless stream of optimization solutions, all of which lead to the same 
goal. \RCC{}\cite{rcc} implemented in NexRes would push throughput beyond single-replica limit by better 
utilizing available resources, opening the door to the development of new high-throughput resilient 
database and federated transaction processing systems.

\par In the presence of Byzantine participants\cite{lao2020survey}, resilience enables blockchains to 
reduce the likelihood of disruption and recover faster. In the business 
world, non-stable system will hugely decrease the chances of converting the visitor into a customer 
and declines customer stickiness. As a consensus-based systems, \RCC{} promises increased resilience against 
failures, can provide strong support for data provenance, and can enable federated data processing in a 
heterogeneous environment with many independent participants. Consequently, consensus-based systems can 
prevent disruption of service due to software issues or cyber attacks that compromise part of the system, 
and can aid in improving data quality of data that is managed by many independent parties, potentially 
reducing the huge societal costs of cyber attacks and bad data.
