\section{Activities and tasks}

\subsection{Client}

\par Clients are the generators of transaction requests. In \RCC{}, the client $\Client{}$ generate a batch 
of client transactions $T$ and send the batch to a preferred replica $\Replica{}$. $Client{}$ sets a timer 
$t_T$ after sending $T$ to $\Replica{}$ and waits for responses from $\f+1$ distinct replicas that claim 
to have comitted $T$. After receving sufficient responses, $\Client{}$ counts the latency of each client 
transaction. $\Client{}$ will complain to other replicas if $\Client{}$ fails to receive $\f+1$ responses 
before $t_T$ is out.

\subsection{Primary}

\par The primary $\Primary{i}$ of instance $\Instance{i}$ is the leading role of consensus in the 
instance. After receiving a client batch $T$, $\Primary{i}$ acts as primary in \PBFT{}, broadcast $T$ 
with instance id $i$ to other replicas.

\par If $\Primary{i}$ is detected to have faulty behavior for the $k$-th time at round $r$, then the proposals 
from $\Primary{i}$ will be ignored by other replicas until round $r + 2^k$.

\subsection{Backup Replica}

\par After receiving a proposal $T$ of instance $\Instance{i}$, a non-faulty replica $\Replica{}$ 
checks the validity of $T$ and if $T$ is proposed by the correct primary $\Primary{i}$. If $T$ is valid, 
then $\Replica{}$ participates in the \PBFT{} agreement process of $T$.

\par If $\Replica{}$ receives a complaint from client $\Client{}$, then it forwards it to the corresponding 
primary $\Primary{}$, sets a timer $t_c$, and waits for a corresponding proposal from $\Primary{}$. If $\Replica{}$ 
fails to receive a corresponding proposal before $t_c$ is out, then $\Primary{}$ is detcted to have faulty behavior. 

\par If $\f+1$ distinct replicas have detected $\Primary{i}$'s faulty behavior, then $\Instance{i}$ will be temporarily 
stalled as mentioned above.

\subsection{Total Ordering}

\par Before executing committed client transactions, \RCC{} totally order the committed transactions between different 
instances. For all committed transactions, \RCC{} execute round by round. In \RCC{} with $\m$ concurrent instances, in 
each round, \RCC{} execute transactions with increasing instance id, from instance $\Instance{0}$ to instance $\Instance{\m-1}$.