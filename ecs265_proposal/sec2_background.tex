\section{Background and Challenges}
The development of consensus-based data processing frameworks is speedy because they provide more resilience during failures and stronger support for data provenance compared with traditional distributed database systems. And consensus protocols are the core of consensus-based systems, which enables independent participants to manage a single common database by reliably and continuously replicating a unique sequence of transactions among all participants. Most practical systems use consensus protocols that follow the classical primary backup design of PBFT. This consensus algorithm works in asynchronous systems and is optimized to be high-performance with an impressive overhead runtime and only a slight increase in latency.\\
Nevertheless, there is an issue of suffering from the limitation of single-replica throughput in traditional primary-backup consensus protocols. To address it, previous work has proposed the idea of concurrent consensus that shows much higher throughput achievement than primary-backup consensus by effectively utilizing all available system resources. Moreover, to push this idea into practice, the RCC paradigm was proposed that turns any primary-backup consensus protocol into a concurrent consensus protocol by running many consensus instances concurrently. The design goals of RCC are:
    \item 1) RCC provides consensus among replicas on the client transactions to be executed and the order in which they are completed.
    \item 2) Clients can interact with RCC to force execution of their transactions and learn the outcome of execution.
    \item 3) RCC is a design paradigm that can be applied to any primary-backup consensus protocol, turning it into a concurrent consensus protocol.
    \item 4) In RCC, consensus-instances with non-faulty primaries are always able to propose transactions at maximum throughput (with respect to the resources available to any replica), this independent of faulty behavior by any other replica.
    \item 5) In RCC, dealing with faulty primaries does not interfere with the operations of other consensus-instances.\\
That is, RCC is designed for maximizing throughput in all cases, even during malicious activity. Furthermore, RCC can make systems more resilient, as it can mitigate the effects of order-based attacks and throttling attacks.\\
With these advantages, RCC has been put to the test by implementing it in \textit{RESILIENTDB}, showing that RCC achieves up to 2.75 times higher throughput than other consensus protocols and can be scaled to 91 replicas.\\
However, the results of the experiments showed that RCC on ResilientDB still has space to improve. Although the results are as expected, it will generate a bottleneck due to the lack of CPU computing capability of RCC implementation. 